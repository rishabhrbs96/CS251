\documentclass{article}

\author{}
\date{}

\begin{document}

\maketitle

\section{Bash}
    \subsection{Introduction}
Bash is a Unix shell and command language written by Brian Fox for the GNU Project as a free software replacement for the Bourne shell. Released in 1989, it has been distributed widely as the shell for the GNU operating system and as a default shell on Linux and OS X. It was announced during the 2016 Build Conference that Windows 10 has added a Linux subsystem which fully supports Bash and other Ubuntu binaries running natively in Windows. In the past, and currently, it has also ported to Microsoft Windows and distributed with Cygwin and MinGW, to DOS by the DJGPP project, to Novell NetWare and to Android via various terminal emulation applications. In the late 1990s, Bash was a minor player among multiple commonly used shells; at present Bash has overwhelming favor.
    \subsection{GREP}
    grep is a command-line utility for searching plain-text data sets for lines matching a \textbf{regular expression}. Grep was originally developed for the Unix operating system, but is available today for all Unix-like systems. Its name comes from the ed command g/re/p (\textbf{g}lobally search a \textbf{re}gular expression and \textbf{p}rint), which has the same effect: doing a global search with the regular expression and printing all matching lines.

\section{Octave}
\section{Latex}
\section{gnuplot}
\section{XFig}
\section{HTML}
\section{Git}
\section{BitBucket}

\end{document}
