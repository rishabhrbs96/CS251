\documentclass{article}

\author{}
\date{}

\begin{document}

\maketitle

\section{Bash}
\section{Octave}
\section{Latex}
\section{gnuplot}
\section{XFig}
Xfig  is a menu-driven tool that allows the user to draw and manipulate
objects interactively under the X Window System.  \emph{ It runs under X
version  } and requires a two- or three-button mouse.
file specifies the name of a file to be edited.   The  objects  in  the
file will be read at the start of xfig.
\\
The figure generated by xfig needs to be post processed by an external tool
to convert to a different, more usable format like JPEG or PNG. This is
usually done with \emph{fig2dev}, a program found in the Transfig package.
\\
XFig was one of the first widely used vector graphics editor (in contrast,
programs like photoshop use raster graphics), which means the images are
essentially stored in forms of bazie curves, basic shapes and straight lines
instead of having separate colors for different pixels. Due to the advent
of other programs like Inkscape which used a lot more of today's available
hardware, the program XFig now belongs as a part of history and is no longer
as popular as it used to be.
\section{HTML}
\section{Git}
\section{BitBucket}

\end{document}
