\documentclass{article}

\author{ Group 22 }
\date{}
\title{Basic Computing Tools}
\begin{document}
	
	\maketitle
	
	\section{Bash}
        \subsection{Introduction}
Bash is a Unix shell and command language written by Brian Fox for the GNU Project as a free software replacement for the Bourne shell. Released in 1989, it has been distributed widely as the shell for the GNU operating system and as a default shell on Linux and OS X. It was announced during the 2016 Build Conference that Windows 10 has added a Linux subsystem which fully supports Bash and other Ubuntu binaries running natively in Windows. In the past, and currently, it has also ported to Microsoft Windows and distributed with Cygwin and MinGW, to DOS by the DJGPP project, to Novell NetWare and to Android via various terminal emulation applications. In the late 1990s, Bash was a minor player among multiple commonly used shells; at present Bash has overwhelming favor.
    \subsection{GREP}
    grep is a command-line utility for searching plain-text data sets for lines matching a \textbf{regular expression}. Grep was originally developed for the Unix operating system, but is available today for all Unix-like systems. Its name comes from the ed command g/re/p (\textbf{g}lobally search a \textbf{re}gular expression and \textbf{p}rint), which has the same effect: doing a global search with the regular expression and printing all matching lines.
        \subsection{SED}
    sed (stream editor) is a Unix utility that parses and transforms text, using a simple, compact programming language. \textbf{sed} was developed from 1973 to 1974 by Lee E. McMahon of Bell Labs, and is available today for most operating systems. sed was based on the scripting features of the interactive editor ed ("editor", 1971) and the earlier qed ("quick editor", 1965–66). sed was one of the earliest tools to support regular expressions, and remains in use for text processing, most notably with the substitution command. Other options for doing "stream editing" include AWK and Perl.

	\section{Octave}
	\section{Latex}
	\LaTeX is a typesetting system that is very suitable for producing scientific and mathematical documents of high typographical quality. It is also suitable for producing all sorts of other documents, from simple letters to complete books.\\
	\\
	\LaTeX enables authors to typeset and print their work at the highest typographical quality, using a predefined, professional layout. \LaTeX was originally written by Leslie Lamport.\\
	
	\LaTeX commands are case sensitive, and take one of the following two formats:
	\begin{enumerate}
		\item They start with a backslash \textbackslash and then have a name consisting of
		letters only. Command names are terminated by a space, a number or
		any other 'non-letter.'
		\item They consist of a backslash and exactly one non-letter.
		\item Many commands exist in a 'starred variant' where a star is appended
		to the command name.
	\end{enumerate}
	\subsection{Input File Structure}
	When \LaTeX processes an input file, it expects it to follow a certain structure. Thus every input file must start with the command.
	\begin{verbatim}
		\documentclass{...}
	\end{verbatim}

This specifies what sort of document you intend to write. After that, add commands to influence the style of the whole document, or load packages that add new features to the \LaTeX system. To load such a package you use the command.	\\
	\begin{verbatim}
		\usepackage{...}
	\end{verbatim}
When all the setup work is done, you start the body of the text with the command.
	\begin{verbatim}
		\begin{document}
	\end{verbatim}
	Now you enter the text mixed with some useful \LaTeX commands. At the end of the document you add the
	\begin{verbatim}
	\end{document}
	\end{verbatim}
	\subsection{Titles, Chapters, and Sections}
	To help the reader find his or her way through your work, you should divide it into chapters, sections, and subsections. \LaTeX supports this with special commands that take the section title as their argument. It is up to you to use them in the correct order.\\
	The following sectioning commands are available for the \textbf{article} class:
	\begin{verbatim}
	\section{...}
	\subsection{...}
	\subsubsection{...}
	\paragraph{...}
	\subparagraph{...}
	\end{verbatim}
	If you want to split your document into parts without influencing the section or chapter numbering use:
	\begin{verbatim}
		\part{...}
	\end{verbatim}
	\LaTeX creates a table of contents by taking the section headings and page
	numbers from the last compile cycle of the document. The command
	\begin{verbatim}
		\tableofcontents
	\end{verbatim}
	The title of the whole document is generated by issuing a command:
	\begin{verbatim}
		\maketitle
	\end{verbatim}
	a footnote is printed at the foot of the current page. Footnotes should always
	be put after the word or sentence they refer to. Footnotes referring to a
	sentence or part of it should therefore be put after the comma or period.
	\begin{verbatim}
	\footnote{footnote text}
	\end{verbatim}
	In scientific publications it is customary to start with an abstract which gives
	the reader a quick overview of what to expect. \LaTeX provides the \textbf{abstract}
	environment for this purpose. Normally \textbf{abstract} is used in documents
	typeset with the article document clas
	\begin{verbatim}
		\begin{abstract}
			The abstract abstract.
		\end{abstract}
	\end{verbatim}
	
\subsection{Tables}
The tabular environment can be used to typeset beautiful tables with
optional horizontal and vertical lines. \LaTeX determines the width of the
columns automatically.\\

Below you can see the simplest working example of a table :

\begin{verbatim}

\begin{center}
\begin{tabular}{ c c c }
cell1 & cell2 & cell3 \\ 
cell4 & cell5 & cell6 \\  
cell7 & cell8 & cell9    
\end{tabular}
\end{center}

\end{verbatim}

\begin{verbatim}
	cell1  cell2  cell3
	cell4  cell5  cell6
	cell7  cell8  cell9
\end{verbatim}
\begin{verbatim}
The tabular environment is more flexible, you can put separator lines in between each column.\\
\end{verbatim}

It was already said that the tabular environment is used to type tables. To be more clear about how it works below is a description of each command.
\begin{verbatim}
{ |c|c|c| }
\end{verbatim}
This declares that three columns, separated by a vertical line, are going to be used in the table. Each c means that the contents of the column will be centred, you can also use r to align the text to the right and l for left alignment. 
\begin{verbatim}
	\hline
\end{verbatim}

This will insert a horizontal line on top of the table and at the bottom too. There is no restriction on the number of times you can use \begin{verbatim}
\hline
\end{verbatim}
. 
\begin{verbatim}
cell1 & cell2 & cell3 \\
Each & is a cell separator and the double-backslash \\ sets the end of this row. 
\end{verbatim}

		\section{gnuplot}
		\section{XFig}
		\section{HTML}
		\section{Git}
		\section{BitBucket}
		
	\end{document}
